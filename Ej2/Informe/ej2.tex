\documentclass[../../main.tex]{subfiles}

\begin{document}

\section{Puente de Wien}

\subsection{Diseño del puente}
\begin{figure}[H]	
	\centering
	\includegraphics[width=0.35\textwidth]{fotos/PuenteGen.png}
	\caption{Puente con impedancias gen\'ericas} \label{fig:pg}
\end{figure}
La tensión de salida del puente de la figura \ref{fig:pg} es $V_d=V_g \frac{Z_3 Z_2 - Z_1 Z_4}{(Z_1 + Z_3)(Z_2 + Z_4)}$. En el equilibrio ($V_d=0$), se cumple que $Z_1 Z_4 = Z_2 Z_3$. En el caso del puente de Wien $Z_1=R_1 + \frac{1}{SC_1}$, $Z_2=R_2$, $Z_3=R_3 +  \frac{1}{SC_3}$ y $Z_4=R_4$. En el equilibrio se cumple que $\frac{C_3}{C_1} \frac{R_1}{R_3}= \frac {R_2}{R_4}$ y $f=\frac{1}{2 \pi \sqrt{R_1 R_3 C_1 C_3} }$. Si $R=R_1=R_3$, $C=C_1=C_3$ y $R_2=2 R_4$ entonces $f=\frac{1}{2 \pi RC}$.

\subsubsection{Elección de componentes}
Asumiendo que  $R=R_1=R_3$, $C=C_1=C_3$ y $R_2=2 R_4$ , se obtuvo que  $f=\frac{1}{2 \pi RC}$. El intervalo de frecuencias que se desea medir es $f \in [10KHZ , 100KHz]$. Fijando $C=820pF$, $R_2=20K\Omega$ y $R_4=10K\Omega$  entonces $R \in \left[ \frac{1}{2 \pi f_{max} C} , \frac{1}{2 \pi f_{min} C} \right]=\left[ 1941 \Omega , 19409 \Omega \right]$. Para conseguir dichos valores de R se utilizó un preset de $25K\Omega $.

\subsubsection{An\'alisis de sensibilidades}
Tal como ya fue mencionado, $R=R_1=R_3$ y dichas resistencias se implementaron con una resistencia de $1.5K\Omega$ en serie con un preset de $25K\Omega$. El present es de 25 vueltas y suponiendo que lo m\'inimo que se puede girar es un cuarto de vuelta, definimos nuestro $\Delta R=250\Omega$. Adem\'as se supuso que la maxima diferencia entre una resistencia de ajuste era $\Delta R=250\Omega$. De esta manera se graficaron las sensibilidades de $V_d$ respecto de $R_1$ y $R_3$ variando $R$ en el rango indicado en la secci\'on anterior. Adem\'as se analizó la sensibilidad de $V_d$ respecto a las mismas resistencias, pero variando la frecuencia en el rango de medición del puente.

\begin{figure}[H]	
	\centering
	\includegraphics[width=0.7\textwidth]{fotos/r3r.png}
	\caption{Sensibilidad de $V_d$ respecto a R3}
\end{figure}

\begin{figure}[H]	
	\centering
	\includegraphics[width=0.7\textwidth]{fotos/r3f.png}
	\caption{Sensibilidad de $V_d$ respecto a R3 variando f}
\end{figure}


\begin{figure}[H]	
	\centering
	\includegraphics[width=0.7\textwidth]{fotos/r1r.png}
	\caption{Sensibilidad de $V_d$ respecto a R1}
\end{figure}

\begin{figure}[H]	
	\centering
	\includegraphics[width=0.7\textwidth]{fotos/r1f.png}
	\caption{Sensibilidad de $V_d$ respecto a R1 variando f}
\end{figure}


Tal como se observa en ambas im\'agenes la sensibilidad de $V_d$ respecto a ambas resistencias es pr\'acticamente la misma, por ende una resistencia no enmascara a la otra. Adem\'as la sensibilidad de ambas resistencias empeora al aumentar la frecuencia.




\subsubsection{Mediciones}
Se midieron las siguientes frecuencias:

\begin{table}[H]
\begin{center}
\begin{tabular}{|c|c|c|c|c|}
\hline
Frecuencia generador$[KHz]$& $R_1 [K\Omega]$ & $R_3[K\Omega]$ & Frecuencia calculada $[KHz]$ &Error[ \% ]\\
\hline \hline
9.7 & 19.4& 19.4&10&3.1  \\ \hline
28.4 & 6.5& 6.5&29.8&5.1  \\ \hline
37.7 & 4.85& 4.85&40&6.1  \\ \hline
54.6 & 3.34& 3.34&58.1&6.4  \\ \hline
75.5 & 2.76& 2.76&70&6.8  \\ \hline
106.5 & 1.94& 1.94&99&8.7  \\ \hline

\end{tabular}
\caption{Mediciones de frecuencias.} 
\end{center}
\end{table}

\subsubsection{Convergencia del puente}
Para analizar la convergencia del puente, se realizó un grafico de $V_d(R,f)$ en matlab variando R y f en los intervalos correspondientes.

\begin{figure}[H]	
	\centering
	\includegraphics[width=0.9\textwidth]{fotos/conv.png}
	\caption{Gr\'afico $V_d(R,f)$}\label{fig:convw}
\end{figure}

Como se observa en la figura \ref{fig:convw}, hay una \'unica franja violeta (m\'inimo). Esto quiere decir, que existe un \'unico valor de R para cada frecuencia f que genera un m\'inimo de $V_d$.

\subsubsection{Conclusión}
Como la convergencia del puente es \'unica para cada valor de resistencia el puente no necesita un manual para su utilizaci\'on. En cuanto al error que obtuvimos en la medici\'on, lo atribuimos a que la medicion se realizón con el osciloscopio y sin amplificador de instrumentacion, adem\'as de a las tolerancias del 10\% en los capacitores. Tambi\'en puede estar entrando en juego el hecho que al aumentar la frecuencia aumenta la sensibilidad frente a las variables de ajuste. Dicho fen\'omeno se observa en el hecho de que el error aumenta con el aumento de la frecuencia.

\end{document}

