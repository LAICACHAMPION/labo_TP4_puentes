

\documentclass[../../main.tex]{subfiles}



\begin{document}
\section{Comparación de los metodos  de medición de equilibrio de puentes}
\subsection*{Osciloscopio}
\par El osciloscopio es un instrumento que no permite visualizar una tensión en el tiempo, por ende nos permite visualizar el punto de equilibrio del puente. Se varían las variables de ajuste de tal manera de que la amplitud de la señal de salida del puente sea lo menor posible. Pero el equipo es sensible al ruido, provocando que sea difícil encontrar el mínimo. Sin embargo, se podría medir en vez de la tensión, el cambio abrupto de fase provocado en el equilibrio, otorgando una mayor precisión.
\subsection*{Multímetro de precisión}
\par En contraposición del osciloscopio, no permite visualizar la señal en el tiempo, sin embargo se puede medir el valor RMS de la señal de salida y variando las variables de ajuste permite encontrar el mínimo de la señal. Como el quipo realiza el valor medio de la señal, es menos susceptible al ruido. Además no se depende del observador para hallar gráficamente el mínimo de la señal, debido a que el multímetro devuelve números. Sin embargo no permite detectar el cambio de fase que ocurre en el equilibrio.
\subsection*{Amplificador de Instrumentación}
\par Por si solo no permite medir el equilibrio, sino que se lo tiene que utilizar con alguno de los equipos ya mencionados. Este dispositivo permite eliminar las señales de modo común que provienen del circuito, por ejemplo el ruido. Por ende permite mejorar la medición. Combinándose con el osciloscopio otorgaría una mejor medición de la tensión de salida y además se podría observar las fase.


\end{document}
